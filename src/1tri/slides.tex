\documentclass[xcolor=dvipsnames]{beamer}

\usepackage{graphicx,subfigure,url}

% example themes
\usetheme{Madrid}
\usecolortheme{UBCBlue}
\usepackage{epigraph}

% put page numbers
% \setbeamertemplate{footline}[frame number]{}
% remove navigation symbols
% \setbeamertemplate{navigation symbols}{}

\title{Coordinació TIC - 1r Trimestre}
\author{Alfons Rovira}

\begin{document}

\frame[plain]{\titlepage}

\begin{frame}[plain]{Continguts}
	\tableofcontents
\end{frame}

\section{Dos camins}
\begin{frame}{El camí no elegit}

  \epigraph{Dos camins es bifurcaben en un bosc, i jo, vaig agafar el menys transitat, i axiò va marcar la diferència.}
{\textit{Robert Frost}}
\end{frame}

\section{Aspects Generals}
\begin{frame}{Aspectes Generals}
  
\begin{itemize}
\item<1->
  Revisió i expansió del \textbf{Pla TIC}.
\item<2->
  Estandarització de \textbf{Lliurex 16}.
\item<3->
  Impuls de la \textbf{gestió de contrasenyes} mitjançant una llibreta analògica
  o un gestor de contrasenyes.
\item<4->
  Adopció i Creació del \textbf{Model d'Aula Lliurex} a la Sala de Reunions.
\item<5->
  Instal·lació i configuració de l'\textbf{App de Moodle}.
\end{itemize}
\end{frame}

\section{Aspectes Pràctics}
\begin{frame}{Aspectes Pràctics}

\begin{itemize}
\item<1->
  Instal·lació i configuració d'un \textbf{client per llegir al correu docent} al
  mòbil.
\item<2->
  Creació d'un \textbf{Bulletí setmanal} d'informació TIC que arriba al correu
  dels docents i que roman a l'aula virtual de Coordinació TIC.
\item<3->
  Creació d'un \textbf{aula virtual} per coordinar els docents de forma virtual.
\item<4->
  Revisió dels ordinadors amb \textbf{3 perfils}:

  \begin{enumerate}
  \item<5->
    TTL
  \item<6->
    HP
  \item<7->
    Inves
  \end{enumerate}
\end{itemize}
\end{frame}


\section{Procediments realitzats a les aules}
\begin{frame}{Procediments realitzats a les aules}

  \begin{enumerate}
  \item<1->
    Revisió i ajust de la \textbf{dimensió correcta del disc}.
  \item<2->
    Configuració de repositori cap al \textbf{mirror del server}.
  \item<3->
    Programació d'\textbf{actualitzacions automàtiques} entre les 10:00 am i les 10:30 am.
  \item<4->
    \textbf{Apagat automàtic} dels ordinadors entre les 15:45 i les 16:00 h.
  \item<5->
    \textbf{Backup automàtic} de les carpetes de Documents, Imatges i Música.
  \item<6->
    Creació d'un \textbf{usuari administrador} amb tots els drets
    \texttt{lliurex-admin} i un usuari d'escriptori, sense drets
    d'administració \texttt{lliurex}.
  \item<7->
    Configuració de la \textbf{pantalla d'inici de Firefox 172.24.23.254}.
  \end{enumerate}
\end{frame}

\section{Configuracions}

\begin{frame}{Configuracions}

\begin{itemize}
\item<1->
  Configuració de l'\textbf{escaner} cap a una carpeta compartida del server
  anomenada \texttt{Profesores}.
\item<2->
  Creació d'un \textbf{backup remot xifrat} del servidor orientat a dades.
\item<3->
  Consulta del disseny dels documents del centre amb el \textbf{nou logotip} a la
  Sotssecretària de la Conselleria d'Educació Cultura i Esport, ens
  respongueren des del Gavient Jurídic citant el Manual d'Imatge
  Corporativa.
\item<4->
  \textbf{Mumble}: Instal·lació i configuració d'un servei de conversa
  d'audio bidireccional que funciona als ordinadors i als Smart
  Phones.Funciona tant amb format d'audio i incorpora un chat.
\item<5->
  \textbf{Plumble}, app per l'Smart Phone.
\end{itemize}
\end{frame}


\section{Aportacions}
\begin{frame}{Aportacions}

\begin{itemize}
\item<1->
  Vam localitzar i configurar fins on es podia l'\textbf{escaner d'ulls TOBI}
  mitjançant software lliure, vam informar al servei de suport de la
  marca, els quals desconeixien aquest software.
\item<2->
  Creació d'un usuari \texttt{projector} per optimitzar la resolució de
  pantalla i el so de l'ordinador que projecta a la Sala de Reunions.
\end{itemize}
\end{frame}

\section{Problemes sorgits}
\begin{frame}{Problemes sorgits}

\begin{itemize}
\item<1->
  L'ordinador a l'aula de fisioteràpia \textbf{falla la targeta de xarxa},
  mirarem la garantia.
\end{itemize}
\end{frame}

\section{Tasques pendents}
\begin{frame}{Tasques pendents}

\begin{itemize}
\item<1->Configuració dels \textbf{punts d'accés Wifi} del centre
  mitjançant un programa de gestió.
\item<2->
  Protocol d'autenticació Wifi mitjançant el servidor, \textbf{Free
  Radious}.
\item<3->
  Al Segon Trimestre \textbf{migrarem els ordinadors HP} a Lliurex 16, cal salvar
  dades abans que vinga el tècnic del SAI.
\item<4->
  Enllestir els \textbf{ordinadors INVES} amb una proposta d'activitats digitals
  interactives orientades als alumnes (JClic, Gcompris, Tux Paint, etc.).
\item<5->
  Desenvolupar un sistema de \textbf{kiosk} per a l'entrada del centre on es
  puguen visualitzar slides o presentacions d'informació del centre
  mitjançant una pantalla mentre s'espera al Hall.
\end{itemize}
\end{frame}

\section{Deures de Nadal}
\begin{frame}{Deures de Nadal}

\begin{itemize}
\item<1->
  \textbf{Certificat Digital de la Conselleria}.
\item<2->
  Certificat Digital \textbf{Cl@ve Pin}.
\item<3->
  Intal·lació de l'\textbf{App Clave Pin}.
\item<4->
  Instal·lació del programari \textbf{Autofirma}, per firmar documents amb
  Certificat Digital
\end{itemize}
\end{frame}

\end{document}